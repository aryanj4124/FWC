\let\negmedspace\undefined
\let\negthickspace\undefined
\documentclass[journal,12pt,twocolumn]{IEEEtran}
\usepackage{cite}
\usepackage{amsmath,amssymb,amsfonts,amsthm}
\usepackage{algorithmic}
\usepackage{graphicx}
\usepackage{textcomp}
\usepackage{xcolor}
\usepackage{txfonts}
\usepackage{listings}
\usepackage{enumitem}
\usepackage{mathtools}
\usepackage{gensymb}
\usepackage[breaklinks=true]{hyperref}
\usepackage{tkz-euclide} % loads  TikW and tkz-base
\usepackage{listings}
\usepackage{gvv}
\usepackage{tfrupee}
%
%\usepackage{setspace}
%\usepackage{gensymb}
%\doublespacing
%\singlespacing

%\usepackage{graphicx}
%\usepackage{amssymb}
%\usepackage{relsize}
%\usepackage[cmex10]{amsmath}
%\usepackage{amsthm}
%\interdisplaylinepenalty=2500
%\savesymbol{iint}
%\usepackage{txfonts}
%\restoresymbol{TXF}{iint}
%\usepackage{wasysym}
%\usepackage{amsthm}
%\usepackage{iithtlc}
%\usepackage{mathrsfs}
%\usepackage{txfonts}
%\usepackage{stfloats}
%\usepackage{bm}
%\usepackage{cite}
%\usepackage{cases}
%\usepackage{subfig}
%\usepackage{xtab}
%\usepackage{longtable}
%\usepackage{multirow}
%\usepackage{algorithm}
%\usepackage{algpseudocode}
%\usepackage{enumitem}
%\usepackage{mathtools}
%\usepackage{tikz}
%\usepackage{circuitikz}
%\usepackage{verbatim}
%\usepackage{tfrupee}
%\usepackage{stmaryrd}
%\usetkzobj{all}
%    \usepackage{color}                                            %%
%    \usepackage{array}                                            %%
%    \usepackage{longtable}                                        %%
%    \usepackage{calc}                                             %%
%    \usepackage{multirow}                                         %%
%    \usepackage{hhline}                                           %%
%    \usepackage{ifthen}                                           %%
  %optionally (for landscape tables embedded in another document): %%
%    \usepackage{lscape}     
%\usepackage{multicol}
%\usepackage{chngcntr}
%\usepackage{enumerate}

%\usepackage{wasysym}
%\documentclass[conference]{IEEEtran}
%\IEEEoverridecommandlockouts
% The preceding line is only needed to identify funding in the first footnote. If that is unneeded, please comment it out.

\newtheorem{theorem}{Theorem}[section]
\newtheorem{problem}{Problem}
\newtheorem{proposition}{Proposition}[section]
\newtheorem{lemma}{Lemma}[section]
\newtheorem{corollary}[theorem]{Corollary}
\newtheorem{example}{Example}[section]
\newtheorem{definition}[problem]{Definition}
%\newtheorem{thm}{Theorem}[section] 
%\newtheorem{defn}[thm]{Definition}
%\newtheorem{algorithm}{Algorithm}[section]
%\newtheorem{cor}{Corollary}
\newcommand{\BEQA}{\begin{eqnarray}}
\newcommand{\EEQA}{\end{eqnarray}}
\newcommand{\define}{\stackrel{\triangle}{=}}
\theoremstyle{remark}
\newtheorem{rem}{Remark}

%\bibliographystyle{ieeetr}
\setlength{\parindent}{0pt}
\begin{document}
\bibliographystyle{IEEEtran}


\vspace{3cm}

\title{
%	\logo{
FWC Assignment-1
%	}
}
\author{ Aryan Jain$^{*}$ FWC22256% <-this % stops a space
	\thanks{}
	
}
%\title{
%	\logo{Matrix Analysis through Octave}{\begin{center}\includegraphics[scale=.24]{tlc}\end{center}}{}{HAMDSP}
%}


% paper title
% can use linebreaks \\ within to get better formatting as desired
%\title{Matrix Analysis through Octave}
%
%
% author names and IEEE memberships
% note positions of commas and nonbreaking spaces ( ~ ) LaTeX will not break
% a structure at a ~ so this keeps an author's name from being broken across
% two lines.
% use \thanks{} to gain access to the first footnote area
% a separate \thanks must be used for each paragraph as LaTeX2e's \thanks
% was not built to handle multiple paragraphs
%

%\author{<-this % stops a space
%\thanks{}}
%}
% note the % following the last \IEEEmembership and also \thanks - 
% these prevent an unwanted space from occurring between the last author name
% and the end of the author line. i.e., if you had this:
% 
% \author{....lastname \thanks{...} \thanks{...} }
%                     ^------------^------------^----Do not want these spaces!
%
% a space would be appended to the last name and could cause every name on that
% line to be shifted left slightly. This is one of those "LaTeX things". For
% instance, "\textbf{A} \textbf{B}" will typeset as "A B" not "AB". To get
% "AB" then you have to do: "\textbf{A}\textbf{B}"
% \thanks is no different in this regard, so shield the last } of each \thanks
% that ends a line with a % and do not let a space in before the next \thanks.
% Spaces after \IEEEmembership other than the last one are OK (and needed) as
% you are supposed to have spaces between the names. For what it is worth,
% this is a minor point as most people would not even notice if the said evil
% space somehow managed to creep in.



% The paper headers
%\markboth{Journal of \LaTeX\ Class Files,~Vol.~6, No.~1, January~2007}%
%{Shell \MakeLowercase{\textit{et al.}}: Bare Demo of IEEEtran.cls for Journals}
% The only time the second header will appear is for the odd numbered pages
% after the title page when using the twoside option.
% 
% *** Note that you probably will NOT want to include the author's ***
% *** name in the headers of peer review papers.                   ***
% You can use \ifCLASSOPTIONpeerreview for conditional compilation here if
% you desire.




% If you want to put a publisher's ID mark on the page you can do it like
% this:
%\IEEEpubid{0000--0000/00\$00.00~\copyright~2007 IEEE}
% Remember, if you use this you must call \IEEEpubidadjcol in the second
% column for its text to clear the IEEEpubid mark.



% make the title area
\maketitle

\newpage

%\tableofcontents

\bigskip

\renewcommand{\thefigure}{\theenumi}
\renewcommand{\thetable}{\theenumi}
%\renewcommand{\theequation}{\theenumi}

%\begin{abstract}
%%\boldmath
%In this letter, an algorithm for evaluating the exact analytical bit error rate  (BER)  for the piecewise linear (PL) combiner for  multiple relays is presented. Previous results were available only for upto three relays. The algorithm is unique in the sense that  the actual mathematical expressions, that are prohibitively large, need not be explicitly obtained. The diversity gain due to multiple relays is shown through plots of the analytical BER, well supported by simulations. 
%
%\end{abstract}
% IEEEtran.cls defaults to using nonbold math in the Abstract.
% This preserves the distinction between vectors and scalars. However,
% if the journal you are submitting to favors bold math in the abstract,
% then you can use LaTeX's standard command \boldmath at the very start
% of the abstract to achieve this. Many IEEE journals frown on math
% in the abstract anyway.

% Note that keywords are not normally used for peerreview papers.
%\begin{IEEEkeywords}
%Cooperative diversity, decode and forward, piecewise linear
%\end{IEEEkeywords}



% For peer review papers, you can put extra information on the cover
% page as needed:
% \ifCLASSOPTIONpeerreview
% \begin{center} \bfseries EDICS Category: 3-BBND \end{center}
% \fi
%
% For peerreview papers, this IEEEtran command inserts a page break and
% creates the second title. It will be ignored for other modes.
%\IEEEpeerreviewmaketitle

\begin{enumerate}
\item Write the direction ratios of the vector $3\vec{a}+2\vec{b}$ where $\vec{a} = \vec{i}+\vec{j}-2\vec{k}$ and $\vec{b} = 2\vec{i}-4\vec{j}+5\vec{k}$. \\
\item Find the projection of the vector $\vec{a}=2\vec{i}+3\vec{j}+2\vec{k}$ on the vector $\vec{b}=2\vec{i}+2\vec{j}+\vec{k}$. \\
\item Write the vector equation of the line passing through $\brak{1,2,3}$ and perpendicular to the plane $\vec{r}\cdot\brak{\vec{i}+2\vec{j}-5\vec{k}}+9=0$. \\
\item In the interval ${\pi}/2<x<\pi$, find the value of $x$ for which the matrix $\myvec{2\sin{x} & 3 \\ 1 & 2\sin{x}}$ is singular.\\
\item Find the solution to the differential equation 
$\frac{dy}{dx} = x^{3}e^{-2y}$
\item Write the integrating factor of the differential equation
$\sqrt{x}\frac{dy}{dx} + y = e^{-2\sqrt{x}}$ \\
\item A trust fund has \rupee 35,000 is to be invested in two different types of bonds. The first bond pays 8\% interest per annum which will be given to orphanage and second bond pays 10\% interest per annum which will be given to an N.G.O. (Cancer Aid Society). Using matrix multiplication, determine how to divide \rupee 35,000 among two types of bonds if the trust fund obtains an annual total interest of \rupee 3,200. What are the values reflected in this question? \\
\item Express the matrix $A = \myvec{2&4&-6\\7&3&5\\1&-2&4}$ as the sum of a symmetric and skew symmetric matrix.
\begin{center} $\vec{OR}$ \\ \end{center}
If $A = \myvec{2&3\\1&-4}, B = \myvec{1&-2\\-1&3}$, verify that $(AB)^{-1} = B^{-1}A^{-1}$. \\
\item Using properties of determinant, solve for x:
$\mydet{a+x & a-x & a-x \\ a-x & a+x & a-x \\ a-x & a-x & a+x} = 0$ \\
\item Evaluate $\int_{0}^{{\pi}/4} \log\brak{1+\tan{x}}dx$. \\
\item Find $\int \frac{x}{\brak{x^2+1}\brak{x-1}}dx$.
\begin{center} $\vec{OR}$ \\ \end{center}
Find $\int_{0}^{\frac{1}{\sqrt{2}}} \frac{\sin^{-1}x}{{\brak{1-x^2}}^{3/2}}dx$.\\
\item Four cards are drawn successively with replacement from a well shuffled deck of 52 cards. What is the probability that
\begin{enumerate}
\item all the four cards are spades?
\item only 2 cards are spades?
\end{enumerate}
\begin{center} $\vec{OR}$ \\ \end{center}
A pair of dice is thrown 4 times, If getting a doublet is considered a success, find the probabbility distribution of number of successes. Hence find the mean of the distribution. \\
\item Prove that $\sbrak{\vec{a},\vec{b}+\vec{c},\vec{d}} = \sbrak{\vec{a},\vec{b},\vec{d}}+\sbrak{\vec{a},\vec{c},\vec{d}}$.  \\
\item Find the shortrest distance between the following lines:\\
$\vec{r} = 2\vec{i}-5\vec{j}+\vec{k} +\lambda\brak{3\vec{i}+2\vec{j}+6\vec{k}}$ and $\vec{r} = 7\vec{i}-6\vec{k}+\mu\brak{\vec{i}+2\vec{j}+2\vec{k}}$. \\
\item Prove that $2\tan^{-1}\brak{\frac{1}{2}}+\tan^{-1}\brak{\frac{1}{7}} = \sin^{-1}\brak{\frac{31}{25\sqrt{2}}}$.
\begin{center} $\vec{OR}$ \\ \end{center}
Solve for x: $\tan^{-1}\brak{\frac{1-x}{1+x}} = \frac{1}{2}\tan^{-1}x$, $x>0$. \\
\item For what value of $\lambda$ the function defined by $f(x) = \begin{cases} \lambda\brak{x^2+2}&, if x \leq 0 \\ 4x+6&, if x>0 \end{cases}$is continuous at $x=0$? Hence check the differentiability of $f(x)$ at $x=0$. \\
\item If $x = ae^t\brak{\sin{t}+\cos{t}}$ and $y = ae^t\brak{\sin{t}-\cos{t}}$, prove that $\frac{dy}{dx}=\frac{x+y}{x-y}$. \\
\item if $y = Ae^{mx} + Be^{nx}$, show that $\frac{d^{2}y}{dx^2} -\brak{m+n}\frac{dy}{dx} + mny = 0$. \\
\item Find $\int \frac{x+3}{\sqrt{5-4x-2x^2}}dx$.\\
\item Show that the relation $R$ in the set $A = \cbrak{1,2,3,4,5}$ given by $R = \cbrak{\brak{a,b} : |a-b| \text{ is divisible by 2}}$ is an equivalence relation. Write all the equivalence classes of $R$. \\ 
\item Find the area of the region $\cbrak{\brak{x,y} : y^2 \leq 4x, 4x^2+4y^2 \leq 9}$, using integration.
\begin{center} $\vec{OR}$ \\ \end{center}
Using integration, find the area enclosed by the parabola $4y = 3x^2$ and the line $2y = 3x+12$. \\
\item Solve the differential equation: \\
$\brak{x\sin^2{\brak{\frac{y}{x}}}-y}dx +xdy = 0$ given $y = \frac{\pi}{4}$ when $x=1$.
\begin{center} $\vec{OR}$ \\ \end{center}
Solve the differential equation $\frac{dy}{dx}-3y\cot{x} = \sin{2x}$ given $y=2$ when $x=\frac{\pi}{2}$.\\
\item Find the vector and cartesian equations of the plane passing through the line of intersection of planes \\
$\vec{r}\cdot\brak{2\vec{i}+2\vec{j}-3\vec{k}}=7$, $\vec{r}\cdot\brak{2\vec{i}+5\vec{j}+3\vec{k}}=9$\\
such that the intercepts made by the plane on x-axis and z-axis are equal.\\
\item In answering a question on a multiple choice test, a student either knows the answer or guesses. Let $\frac{3}{5}$ be the probability that he knows the answer and $\frac{2}{5}$ be the probability that he guesses. Assuming that a student who gusses at the answer will be correct with probability $\vec{1}{3}$, what is the probability that the student knows the answer given that he answered it correctly?\\
\item A manufacturer produces nuts and bolts. It takes 2 hours work on machine A and 3 hours work on machine B to produce a package of nuts. It takes 3 hours on machine A and 2 hours on machine B to produce a package of bolts. He earns a profit of \rupee 24 per package on nuts and \rupee 18 per package on bolts. How many packages of each should be produced each day so as to maximize his profit, if he operates his machines for at the most 10 hours a day. Make an L.P.P. from above and solve it graphically.\\
\item The sum of surface areas of a spere and a cuboid with sides $\frac{x}{3}$, $x$ and $2x$ is constant . Show that the sum of their volumes is minimum if $x$ i equal to three times the radius of sphere.
\end{enumerate}
\end{document}
