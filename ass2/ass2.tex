\documentclass[10pt,-letter paper]{article}
\usepackage[left=1in, right=0.75in, top=1in, bottom=0.75in]{geometry}
\usepackage{graphicx} % Required for inserting images
\usepackage{siunitx}
\usepackage{setspace}
\usepackage{gensymb}
\usepackage{xcolor}
\usepackage{caption}
%\usepackage{subcaption}
\doublespacing
\singlespacing
\usepackage[none]{hyphenat}
\usepackage{amssymb}
\usepackage{relsize}
\usepackage[cmex10]{amsmath}
\usepackage{mathtools}
\usepackage{amsmath}
\usepackage{commath}
\usepackage{amsthm}
\interdisplaylinepenalty=2500
%\savesymbol{iint}
\usepackage{txfonts}
%\restoresymbol{TXF}{iint}
\usepackage{wasysym}
\usepackage{amsthm}
\usepackage{mathrsfs}
\usepackage{txfonts}
\let\vec\mathbf{}
\usepackage{stfloats}
\usepackage{float}
\usepackage{cite}
\usepackage{cases}
\usepackage{subfig}
%\usepackage{xtab}
\usepackage{longtable}
\usepackage{multirow}
%\usepackage{algorithm}
\usepackage{amssymb}
%\usepackage{algpseudocode}
\usepackage{enumitem}
\usepackage{mathtools}
%\usepackage{eenrc}
%\usepackage[framemethod=tikz]{mdframed}
\usepackage{listings}
%\usepackage{listings}
\usepackage[latin1]{inputenc}
%%\usepackage{color}{   
%%\usepackage{lscape}
\usepackage{textcomp}
\usepackage{titling}
\usepackage{hyperref}
%\usepackage{fulbigskip}   
\usepackage{circuitikz}
\usepackage{graphicx}
\lstset{
  frame=single,
  breaklines=true
}
\let\vec\mathbf{}
\usepackage{enumitem}
\usepackage{graphicx}
\usepackage{siunitx}
\let\vec\mathbf{}
\usepackage{enumitem}
\usepackage{graphicx}
\usepackage{enumitem}
\usepackage{tfrupee}
\usepackage{amsmath}
\usepackage{amssymb}
\usepackage{mwe} % for blindtext and example-image-a in example
\usepackage{wrapfig}
\graphicspath{{figs/}}
\providecommand{\cbrak}[1]{\ensuremath{\left\{#1\right\}}}
\providecommand{\brak}[1]{\ensuremath{\left(#1\right)}}
\newcommand{\sgn}{\mathop{\mathrm{sgn}}}
\providecommand{\abs}[1]{\left\vert#1\right\vert}
\providecommand{\res}[1]{\Res\displaylimits_{#1}} 
\providecommand{\norm}[1]{\left\lVert#1\right\rVert}
%\providecommand{\norm}[1]{\lVert#1\rVert}
\providecommand{\mtx}[1]{\mathbf{#1}}
\providecommand{\mean}[1]{E\left[ #1 \right]}
\providecommand{\fourier}{\overset{\mathcal{F}}{ \rightleftharpoons}}
%\providecommand{\hilbert}{\overset{\mathcal{H}}{ \rightleftharpoons}}
\providecommand{\system}{\overset{\mathcal{H}}{ \longleftrightarrow}}
	%\newcommand{\solution}[2]{\textbf{Solution:}{#1}}
%\newcommand{\solution}{\noindent \textbf{Solution: }}
\newcommand{\cosec}{\,\text{cosec}\,}
\providecommand{\dec}[2]{\ensuremath{\overset{#1}{\underset{#2}{\gtrless}}}}
\newcommand{\myvec}[1]{\ensuremath{\begin{pmatrix}#1\end{pmatrix}}}
\newcommand{\myaugvec}[2]{\ensuremath{\begin{amatrix}{#1}#2\end{amatrix}}}
\newcommand{\mydet}[1]{\ensuremath{\begin{vmatrix}#1\end{vmatrix}}}
\date{\today}
\begin{document}
\author{Aryan Jain$^{*}$ FWC22256}
\title{Gate IN2018, 44}
\date{\today}
\maketitle
\begin{enumerate}
\item A 2-bit synchronous counter using two J-K flip flops is shown. The expression for the inputs to the J-K flip flops are also shown in the figure. The output sequence of the counter starting from $Q_{1}Q_{2} = 00$ is
\hfill{GATE-IN2018,44}
\end{enumerate}
\begin{figure}[!ht]
\centering
\begin{circuitikz}
\tikzstyle{every node}=[font=\small]
\draw [short] (3.75,13.5) -- (3.75,11);
\draw [short] (3.75,11) -- (3.75,9.75);
\draw [short] (3.75,13.5) -- (6.25,13.5);
\draw [short] (6.25,13.5) -- (6.25,9.75);
\draw [short] (3.75,9.75) -- (6.25,9.75);
\draw [short] (12.5,13.5) -- (12.5,9.75);
\draw [short] (12.5,9.75) -- (15,9.75);
\draw [short] (15,9.75) -- (15,13.5);
\draw [short] (12.5,13.5) -- (15,13.5);
\draw [short] (3.75,12) -- (4,11.75);
\draw [short] (4,11.75) -- (3.75,11.5);
\draw [short] (12.5,12) -- (12.75,11.75);
\draw [short] (12.75,11.75) -- (12.5,11.5);
\draw[] (3.75,11.75) to[short] (1.25,11.75);
\draw [](6.25,13) to[short] (7.5,13);
\draw [](2.5,13) to[short] (3.75,13);
\draw [](3,10.5) to[short] (3.75,10.5);
\draw [](11.25,13) to[short] (12.5,13);
\draw [](11.5,10.25) to[short] (12.5,10.25);
\node [font=\small] at (4.25,13) {$J$};
\node [font=\small] at (4.25,10.5) {$K$};
\node [font=\small] at (5.75,13) {$Q$};
\node [font=\small] at (5.75,10.5) {$\overline{Q}$};
\node [font=\small] at (5,13.25) {SET};
\node [font=\small] at (5,10) {CLR};
\node [font=\small] at (8,13) {$Q_1$};
\node [font=\small] at (1.75,13) {$Q_{1}+Q_{2}$};
\node [font=\small] at (2,10.5) {$\overline{Q_1}+\overline{Q_2}$};
\draw[] (1.25,11.75) to[short] (0.75,11.75);
\draw [](0.75,11.75) to[short] (0.75,7.25);
\draw [](0.75,7.25) to[short] (1.5,7.25);
\draw [](0,7.25) to[short] (0.75,7.25);
\node [font=\small] at (-1,7.25) {Clock};
\node [font=\small] at (13,13) {$J$};
\node [font=\small] at (13,10.25) {$K$};
\node [font=\small] at (14.5,13) {$Q$};
\node [font=\small] at (14.5,10.25) {$\overline{Q}$};
\node [font=\small] at (13.75,13.25) {SET};
\node [font=\small] at (13.75,10) {CLR};
\node [font=\small] at (10,13) {$\overline{Q_1}+Q_2$};
\node [font=\small] at (10.25,10.25) {$Q_1+\overline{Q_2}$};
\draw (0,7.25) to[short] (0.75,7.25);
\draw [](15,13) to[short] (15.5,13);
\node [font=\small] at (15.75,13) {Q2};
\draw [](1.5,7.25) to[short] (8.5,7.25);
\draw [](8.5,7.25) to[short] (8.5,11.25);
\draw [](8.5,11.25) to[short] (8.5,11.5);
\draw[] (12.5,11.75) to[short] (8.5,11.75);
\draw [](8.5,11.75) to[short] (8.5,11.5);
\end{circuitikz}


\label{fig:block_diagram}
\end{figure}
\begin{enumerate}[label=\Alph*.]
\item $00 \rightarrow 11 \rightarrow 10 \rightarrow 01 \rightarrow 00 \hdots $
\item $00 \rightarrow 01 \rightarrow 10 \rightarrow 11 \rightarrow 00 \hdots $
\item $00 \rightarrow 01 \rightarrow 11 \rightarrow 10 \rightarrow 00 \hdots $
\item $00 \rightarrow 10 \rightarrow 11 \rightarrow 01 \rightarrow 00 \hdots $
\end{enumerate}
\end{document}
